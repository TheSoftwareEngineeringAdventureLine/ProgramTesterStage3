% !TEX root = SystemTemplate.tex

\chapter{System  and Unit Testing}

Testing was first done on small parts of the program, and as we added more, we tested larger parts of the program.

\section{Overview}
In general, we began testing on a lower level by testing our 
capabilities for rerouting input and output. Once we managed that,
we moved on to comparing output files to the test case files. Once Dr. 
Logar released the test cases, we created a function to confirm directory 
traversals. Then, when we combined the two, our program functioned correctly
and we were able to add the finishing touches.



\section{Dependencies}
No dependencies.


\section{Test Setup and Execution}
\subsection{Studenfile:///C:/Users/Ben/SkyDrive/Documents/2014 Spring/SoftEng/Sprint2/SystemDiagram.vsdxt Grading}
The student grading section involves
\begin{itemize}
\item Crawling through student directories.
\item Running each students source code on the test cases.
\end{itemize}
The product was tested on the customer supplied example class directory and class directories where created to for further testing.

\paragraph{} Below is a list of the individual directories that where used in testing as well as the difference between them that tested aspects of the product.

\paragraph{Customer Test CSC\_150 Class Directory}
This product test demonstrates the products ability to test a class without any critical test cases multiple directories. It also demonstrates automatic test case generation.
\begin{verbatim}
Customer Test CSC_150 Class Directory:
  Students:
    bad_student_1 Fails some test cases
    bad_student_2 Fails some test cases
    student_3 Passes all test cases
    student_4 Passes all test cases
    
  Multiple directories containing tests
  No critical test cases
  Golden Source Code: average.cpp
  directories containing useless files
\end{verbatim}

\paragraph{Test Directory 1}
This created test demonstrates the products ability to fail a student if they do not pass a critical test case. It also demonstrates automatic test case generation.
\begin{verbatim}
Test Directory 1:
  Students:
    Alex_Johnson
    Bob
    Francis
    John
    
  Critical and normal test cases
  Golden Source Code: max_3.cpp
\end{verbatim}

\paragraph{Test Directory 2}
This created test demonstrates the programs ability fail a student if they do not pass a critical test case; it demonstrates the ability of the product to find test cases in subdirectories; finally it demonstrates the products ability to allow test case generation only when a golden source code is available.
\begin{verbatim}
Test Directory 2:
  Students:
    student_1 Fails critical
    student_2 Fails critical
    student_3 Passes all test cases
    student_4 Passes all test cases
    student_5 Passes all test cases
    student_6 Passes all test cases
    student_7 Passes all test cases

  Test case subdirectories
  Critical and normal test cases
  Golden Source Code: none
\end{verbatim}


