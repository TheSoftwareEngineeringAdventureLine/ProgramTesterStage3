% !TEX root = SystemTemplate.tex

\chapter{System and Unit Testing}

Testing was first done on individual function, or processes if a process required more than one function, and then on larger parts of the program as our individual peices were integrated. 

\section{Overview}
In general, we began testing on our directory traversal and determining which type of subdirectory we had encountered. Once we could filter out the source code and test/answer files we tested our ability to change into these directories and perform the proper actions, compilation for source code and testing the student's program for test files. After we were able to do all this we began testing our log files and summary file and adjusted the foramtting as needed. Finally, once the program was working, we implemented and tested running the test generation code and critical test cases as a part of the program, rather than individually. 


\section{Dependencies}
No dependencies other than a traversable directory be specified on the command line.  


\section{Test Setup and Execution}
\subsection{Student Grading}
The student grading section involves
\begin{itemize}
\item Crawling through student directories.
\item Running each students source code on the test cases.
\item Generating test cases.
\end{itemize}
The product was tested on the customer supplied example class directory and class directories where created to for further testing.

\paragraph{} Below is a list of the individual directories that were used in testing as well as the difference between them that tested aspects of the product.

\paragraph{Customer Test CSC\_150 Class Directory}
This product test demonstrates the products ability to test a class without any critical test cases and multiple directories. It also demonstrates automatic test case generation.
\begin{verbatim}
Customer Test CSC_150 Class Directory:
  Students:
    bad_student_1 Fails some test cases
    bad_student_2 Fails some test cases
    student_3 Passes all test cases
    student_4 Passes all test cases
    
  Multiple directories containing tests
  No critical test cases
  Golden Source Code: average.cpp
  directories containing useless files
\end{verbatim}

\paragraph{Test Directory 1}
This created test demonstrates the products ability to fail a student if they do not pass a critical test case. It also demonstrates automatic test case generation.
\begin{verbatim}
Test Directory 1:
  Students:
    Alex_Johnson
    Bob
    Francis
    John
    
  Critical and normal test cases
  Golden Source Code: max_3.cpp
\end{verbatim}

\paragraph{Test Directory 2}
This created test demonstrates the programs ability fail a student if they do not pass a critical test case; it demonstrates the ability of the product to find test cases in subdirectories; finally it demonstrates the products ability to allow test case generation only when a golden source code is available.
\begin{verbatim}
Test Directory 2:
  Students:
    student_1 Fails critical
    student_2 Fails critical
    student_3 Passes all test cases
    student_4 Passes all test cases
    student_5 Passes all test cases
    student_6 Passes all test cases
    student_7 Passes all test cases

  Test case subdirectories
  Critical and normal test cases
  Golden Source Code: none
\end{verbatim}

\paragraph{Test Generation Function}
Multiple test cases were run against the function that generates the random values for the generated .tst files. Some of these test scenarios are:
\begin{itemize}
    \item Passing in different values when either "int" or "float" are selected initially (typical testing)
    \item Passing in floating minimum and maximum values to integer generation. 
    \item Intentional use of improper values.
    \item Generating test cases in a directory where they had previously been stored (will not overwright old files).
\end{itemize}

Although we have extensively tested this code, there are a few errors that were not handled (explained in this function's code documentation). These errors were not handled as they are "special cases" and would sacrifice the readibility of the function. 
