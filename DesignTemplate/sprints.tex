% !TEX root = SystemTemplate.tex

\chapter{Sprint Reports}

\section{Sprint Report \#1}
Intentionally left blank (there was no sprint report by Latex Samurai).


\section{Sprint Report \#2}

At this point there is only one sprint. So far, our product seems to be functioning to customer specifications. There are no known bugs
in our project related to Assignment Grader.
It meets
each of the requirements outlined by the user stories. It does do "good practice" error checking on user input.
For a second release version, this meets everything that it is required by the customer, Dr. Logar. 
\par{} This sprint
was differnent in that most of the time was spent in the testing phase. We believe this was due to the heightend complexity of the product. Also a lot of time was lost on consolidation of each team members code in to one working product.
In the interest of the future of the product, the members of Kernel Panic will remain in their current
positions for the next sprint.

\section{Sprint Reports for Team \#3 - The Adventure Line by Jonathan Tomes}
\subsection{Sprint 1}
Forgot to do this back on sprint 1, so I'll do a retrospective on sprint 1 now at the end of sprint 3.

Overall sprint 1 went off with out a hitch. Everyone did their parts and contributed well to the project.

\subsection{Sprint 2}
Had to redo a lot of the code. Most of it wasn't split into separate functions and some of it was
improperly documented. After getting around that, we got it to traverse the root directory, finding the
student directories and testing them against the tests located in the test directory. It can generate random
tests (the number of and data types specified by the user). Introduced a simple menu system to make it easier
to generate then run tests. It will log the results of the testing each student into a student log located in their
directory, and to a class log located in the root directory.

\subsection{Sprint 3}
This sprint was a complete reversal of sprint 2's code. This time there were FAR too many functions in
the code that we got. It was extremely difficult to work out how the original code worked. Modifying the
existing code was extremely difficult. I feel that myself and the other members were kind of drained after
the second sprint, along with all the other projects in our other classes. There were several problems with
the code that we received. Poor in-line commenting and explanation of logic; poor naming conventions in both
function names and variables; functions being called with out any explanation as to why it was being called
or what it was expecting to get back; etc. Several example functions of such are: usage, err\_usage, generateFiles,
isGolden, run\_file, test\_loop, test\_code. 

Sadly because of these difficulties, we were unable to get all of
the desired functionality working with sprint 3. I'm a little disappointed in my self for not starting sooner
to find these problems sooner.. but I'm unsure how much that would have really helped. There would have been
no way to integrate our code from sprint 2 into the code we received. So short of having deciding to scrap
the code we got... I do not know if there is anything that could have been gained by starting sooner. As
for why we did not scrap the code that we got, we were unsure if that was even a valid option. For if this
was supposed to be based on working in a company, if we scraped the code we would have received from the company
we would have had to start over from scratch.

In short, this sprint could have gone a lot better, but we learned a few things from it.