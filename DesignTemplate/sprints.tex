% !TEX root = SystemTemplate.tex

\chapter{Sprint Reports}

\section{Sprint Report \#1}
Intentionally left blank (there was no sprint report by Latex Samurai).


\section{Sprint Report \#2}

At this point there is only one sprint. So far, our product seems to be functioning to customer specifications. There are no known bugs
in our project related to Assignment Grader.
It meets
each of the requirements outlined by the user stories. It does do "good practice" error checking on user input.
For a second release version, this meets everything that it is required by the customer, Dr. Logar. 
\par{} This sprint
was differnent in that most of the time was spent in the testing phase. We believe this was due to the heightend complexity of the product. Also a lot of time was lost on consolidation of each team members code in to one working product.
In the interest of the future of the product, the members of Kernel Panic will remain in their current
positions for the next sprint.

\section{Sprint Reports for Team \#3 - The Adventure Line by Jonathan Tomes}
\subsection{Sprint 1}
Forgot to do this back on sprint 1, so I'll do a retrospective on sprint 1 now at the end of sprint 3.

Overall sprint 1 went off with out a hitch. Everyone did their parts and contributed well to the project.

\subsection{Sprint 2}
Had to redo a lot of the code. Most of it wasn't split into seperate functions and some of it was
misdocumented. After getting around that, we got it to travese the root directory, finding the
student directories and testing them against the tests located in the test directory. It can generate random
tests (the number of and data types specified by the user). Introduced a simple menu system to make it easier
to generate then run tests. It will log the results of the testing each student into a student log located in their
directory, and to a class log located in the root directory.

\subsection{Sprint 3}
This sprint was a complete reversal of sprint 2's code. This time there were FAR too many functions in
the code that we got. It was extremely difficult to work out how the original code worked. Modifying the
existing code was extremely difficult. I feel that myself and the other members were kind of drained after
the second sprint, along with all the other projects in our other classes. After a slow start though we
managed to pull out this sprint in time and got it working after a few late nights.