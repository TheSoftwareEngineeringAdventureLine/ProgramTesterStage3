% !TEX root = SystemTemplate.tex
\chapter{User Stories, Backlog and Requirements}
\section{Overview}


This section covers user stories, backlog and requirements for the system.  





\subsection{Scope}

This document contains stakeholder information, 
initial user stories, requirements, proof of concept results, and various research 
task results. 



\subsection{Purpose of the System}
The purpose of the product is to grade many students' {\tt <filename>.cpp} file by running test files and comparing the results to answer files, and assigning percentage grade.


\section{ Stakeholder Information}


This section would provide the basic description of all of the stakeholders for 
the project.

\subsection{Customer or End User (Product Owner)}
Benjamin Sherman is the product owner in this project, who is in contact with the scrum master and technical lead regarding the backlog. 

\subsection{Management or Instructor (Scrum Master)}
James Tillma is the scrum master, who breaks the project into smaller tasks, and is in touch with both product owner and technical lead.

%\subsection{Investors}
%Are there any?  Who?  What role will they play? 


\subsection{Developers --Testers}
Anthony Morast is the technical lead for Sprint 1, and is in contact with both Tillma and Sherman regarding the requirements during scrum meetings and through Trello notes. 



\section{Business Need}
This product is essential for grading computer science programs focused on numerics. All the user has to do is have test cases and expected results in the directory that the {\tt <filename>.cpp} file is in and any of the subdirectories, and run the {\tt grade.cpp} program. It saves a lot of time, and is efficient. 

\section{Requirements and Design Constraints}
Use this section to discuss what requirements exist that deal with meeting the 
business need.  These requirements might equate to design constraints which can 
take the form of system, network, and/or user constraints.  Examples:  Windows 
Server only, iOS only, slow network constraints, or no offline, local storage capabilities. 


\subsection{System  Requirements}
This product runs on the Linux machines in the Opp Lab. 


\subsection{Network Requirements}
This software does not require an internet connection of any sort. 


\subsection{Development Environment Requirements}
There are not any development environment requirements except that there must be a C++ compiler.


\subsection{Project  Management Methodology}
The method used to manage this project is {\tt scrum}. The scrum master met with the product owner, and broke the tasks down to the technical lead. The team meets for ten minutes long scrum meetings to go over the progress, next steps, and impediments. 
 
\begin{itemize}
\item Trello is used to keep track of the backlogs and sprint status
\item Everyone has access to the Sprint and Product Backlogs
\item This project will take three Sprints
\item Each Sprint is two weeks long
\item There are no restrictions on source control 
\end{itemize}

\section{User Stories}
This section contains the user stories regarding functional requirements and how the team broke them down.


\subsection{User Story \#1}:
As a professor I want to be able to automatically grade a student's source code so that I can save time.

\subsubsection{User Story \#1 Breakdown} 
The highlight of this story is "automatically". The professor does not want to have to interact with the grading tool after it is run.
\subsection{User Story \#2} 

As a professor I want to be able to provide test cases with answers so that I can grade different program assignments.

\subsubsection{User Story \#2 Breakdown}
This story refers to assembling test cases and running the student code on those test cases

\subsection{User Story \#3} 

As a professor I want to be able to regrade a students source code without loosing the data from a previous
grading so that I can see what changes upgrading test cases have.

\subsubsection{User Story \#3 Breakdown}
This story refers to saving data. Data from an old runtime should not be removed by a new runtime. Data should be appended to a neatly formatted file for viewing.

\subsection{User Story \#4}
I want to be able to run this program on an entire class of students.

\subsubsection{User Story \#4 Breakdown}
The highlight of this story is that the user wants to be able to run the program one time for an entire class of students. This will require more directory crawling.

\subsection{User Story \#5}
I as a user want to be able to generate test cases within a certain range instead of being able to only use existing test cases.

\subsubsection{User Story \#5 Breakdown}
The important part of this user story is generating test cases. The user will need to provide a min/max, a type of int/float, a number of cases to generate per file, and a number of files to generate. This will be the only portion requiring user input beyond command arguments.

\subsection{User Story \#6}
I as a user want to be able to be able to make a certain test case critical. Meaning, that test case will make the student's program a complete failure if their program does not process it properly.

\subsubsection{User Story \#6 Breakdown}
The highlight of this story is the potential for a student to "critically fail" a program. That is, it my no longer matter how their code does on other test cases because if it fails a critical one, it instantly fails.

