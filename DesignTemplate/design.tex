% !TEX root = SystemTemplate.tex
\chapter{Design  and Implementation}
 This section describes the design details for the overall system as well as individual major components. As a user, you will have a good understanding of the implemetation details without having to look into the code. Note that the
 code to generate test cases is only run if there is a -g cammond line switch is specified after the directory to test is entered.

 Here is an overview of the algorithm:

\begin{itemize}
  \item Determine if there is a .cpp file in the root directory, if so it is used to generate test cases
  \item Create a vector of every subdirectory in program folder
  \item Change to directory where student subdirectories are located
  \item Step through each directory in the directory vector and determine if it contains tests or student source code
  \item Create a vector containing the name each student's source and one containing the path to their source code
  \item Create a vector of all the test cases
  \item While the source code vector is not empty
  \item Compile the program
  \item Determine if the student passes the critical test cases, if not stop testing
  \item Run code against tests in the test vecotor
  \item Count whether the program passed or failed test case
  \item Change back to home directory (where program is located)
  \item Create a queue of every .tst file in home directory
  \item While test case queue is not empty:
  \item Dequeue first test case in queue
  \item Run program using that test case
  \item Count whether the program passed or failed test case
  \item Write log file containing percentage of tests passed and final grade
  \item Write students results to summary file
\end{itemize}


\section{Traversing Subdirectories }

\subsection{Technologies  Used}
The dirent.h library is used for traversing subdirectories.

\subsection{Design Details}

\begin{lstlisting}

bool change_dir(string dir_name)
{
    string path;
    if(chdir(dir_name.c_str()) == 0) 
    {
        path = get_pathname();
        return true;
    }
    return false;
}

bool is_dir(string dir)
{
    struct stat file_info;
    stat(dir.c_str(), &file_info);
    if ( S_ISDIR(file_info.st_mode) ) 
        return true;
    else 
        return false;
}
\end{lstlisting}

\section{Running the Program Using Test Cases }

\subsection{Technologies  Used}
The software was designed in the Linux Environment.


\subsection{Design Details}


\begin{lstlisting}
int run_file(string cpp_file, string test_case) //case_num
{
    //create .out file name
    string case_out(case_name(test_case, "out"));

    //set up piping buffers
    string buffer1("");
    string buffer2(" &>/dev/null < ");
    string buffer3(" > ");

    // "try using | "
    //construct run command, then send to system
    //./<filename> &> /dev/null  < case_x.tst > case_x.out
    buffer1 += cpp_file + buffer2 + test_case + buffer3 + case_out;
    system(buffer1.c_str());

    //0 = Fail, 1 = Pass
    return result_compare(test_case);
}
\end{lstlisting}


